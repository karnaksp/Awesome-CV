%-------------------------------------------------------------------------------
%	SECTION TITLE
%-------------------------------------------------------------------------------
\cvsection{Work Experience}


%-------------------------------------------------------------------------------
%	CONTENT
%-------------------------------------------------------------------------------
\begin{cventries}

%---------------------------------------------------------
  \cventry
    {Data Engineer} % Job title
    {Axenix} % Organization
    {Санкт-Петербург, Россия} % Location
    {Авг. 2024 - Окт. 2025} % Date(s)
    {
      \begin{cvitems}
        \item {Участие в Big Tech проектах банковского сектора в архитектуре \textbf{Data Factory}.}
        \item {Работа в распределённой экосистеме данных, объединяющей десятки доменов и команд, в условиях высокой нагрузки и требований к масштабируемости.}
        \item {Разработка и поддержка ядра хранилища данных (DWH) в рамках \textbf{лейкхаус-архитектуры}, обеспечение согласованности, качества и доступности данных для аналитики (\textbf{Greenplum}).}
        \item {Интеграция данных из разнородных источников: \textbf{data lakes} (S3, HDFS, Hive, Delta Lake), аналитических платформ (ClickHouse), data platforms и других систем.}
        \item {Формирование \textbf{ETL/ELT конвейеров} передачи данных с использованием декларативного подхода к динамической генерации DAG’ов для Airflow (YAML + Python).}
        \item {Поддержка и развитие существующей архитектуры \textbf{Data Vault 2.0}.}
        \item {Разработка в долговременном релизном цикле на разных средах (от dev до prod), выполнение hotfix.}
        \item {Мониторинг потоков данных (Grafana, Prometheus), контроль их качества и полноты, оперативное устранение ошибок путём выполнения DataFix.}
        \item {Разработал модульную систему динамической генерации YAML-конфигураций для DAG’ов на основе Excel-маппингов, что позволило сократить время настройки новых ETL-процессов на 40\% (Python).}
        \item {Создал комплексный набор инструментов автоматизации разработки: парсер логов Airflow (PowerShell), генератор синтетических данных, инструменты обработки артефактов и взаимодействия с микросервисами (Python), а также командную оболочку для управления Git-процессами (bash), что ускорило адаптацию новых сотрудников и уменьшило количество дефектов на code-review.}
        \item {Интегрировал в процесс командной разработки \textbf{code-review}, peer-review и pre-push проверки.}
        \item {Оптимизировал генератор ETL-потоков совместно с коллегами из других команд: внедрил шаблонизацию, асинхронную обработку и расширил покрытие генерируемых артефактов.}
        \item {Автоматизировал процесс подготовки артефактов для датафиксов на PL/pgSQL (проверки на согласованность, ожидаемые результаты), сократив время согласования DataFix на 80\%.}
        \item {Создавал и поддерживал хранимые процедуры, функции и представления в \textbf{Greenplum}.}
        \item {Организовал междоменное сообщество разработчиков в Telegram для обмена инструментами и лучшими практиками.}
      \end{cvitems}
    }


%---------------------------------------------------------
  \cventry
    {Data Scientist} % Job title
    {Spacecode} % Organization
    {Краснодар, Россия (удаленно)} % Location
    {Апр. 2024 - Авг. 2024} % Date(s)
    {
      \begin{cvitems}
        \item {Разработал \textbf{мультимодального LLM-ассистента} с использованием RAG (LangChain, Unstructured, HuggingFace, Llama.cpp/vLLM), построив архитектуру хранения на \textbf{ChromaDB} (горячее хранилище) и \textbf{S3} (холодное хранилище).}
        \item {Построил систему \textbf{динамического ценообразования} на основе интерпретируемых моделей; генерация фич осуществлялась с помощью \textbf{PySpark} из сырых слоёв DWH и Data Lake.}
        \item {Оркестрировал \textbf{ETL/ML-пайплайны} через \textbf{Apache Airflow}, интегрируя запуск Spark-джобов на кластере и контроль качества данных.}
        \item {Внедрил \textbf{DVC} для версионирования датасетов и моделей, связав его с артефактами Spark и метаданными из Airflow.}
        \item {Настроил \textbf{DVC + Git workflow} для управления артефактами ML, что сократило время отладки и аудита на 70\%.}
      \end{cvitems}
    }


%---------------------------------------------------------
  \cventry
    {Data Analyst} % Job title
    {АСК «ДиректСервис»} % Organization
    {Новосибирск, Россия (удаленно)} % Location
    {Сен. 2023 - Янв. 2025} % Date(s)
    {
      \begin{cvitems}
        \item {Реализовал с нуля \textbf{enterprise-level Data Warehouse (DWH)} на масштабах десятков терабайт данных, используя \textbf{PostgreSQL} как основное хранилище и \textbf{ClickHouse} как аналитическое.}
        \item {Автоматизировал процесс обновления витрин данных, создав комплексный пайплайн на основе \textbf{Shell-скриптов} и \textbf{Windows Task Scheduler}.}
        \item {Настроил механизм миграции данных из \textbf{ClickHouse} в \textbf{PostgreSQL}, обеспечив бесперебойную работу аналитических систем.}
        \item {Совместно с системными аналитиками разработал методологии мониторинга и оптимизации производительности DWH.}
        \item {Настроил мониторинг на основе расширений \textbf{Postgres} и внедрил \textbf{CI/CD} для верифицированной доставки кода с использованием \textbf{Flyway} и \textbf{GitHub Actions}.}
      \end{cvitems}
    }


%---------------------------------------------------------
  \cventry
    {ML Engineer} % Job title
    {ООО «ColorDent»} % Organization
    {Москва, Россия (удаленно)} % Location
    {Июн. 2023 - Дек. 2023} % Date(s)
    {
      \begin{cvitems}
        \item {Разработал \textbf{ML-классификатор изображений зубов} для диагностики заболеваний, используя \textbf{OpenCV} и \textbf{TensorFlow Lite}. Модель достигла точности более 95\% на тестовых данных.}
        \item {Интегрировал ML-модели в мобильное приложение с использованием \textbf{MLflow} для управления версиями моделей и \textbf{DVC} для версионирования данных.}
        \item {Оптимизировал производительность модели для работы на устройствах с ограниченными ресурсами, что позволило использовать её на мобильных платформах без значительной потери качества.}
        \item {Разработал систему отслеживания метрик обучения моделей и внедрил автоматическое тестирование для обеспечения высокого качества решения.}
      \end{cvitems}
    }


%---------------------------------------------------------
\end{cventries}


