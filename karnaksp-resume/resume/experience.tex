%-------------------------------------------------------------------------------
%	SECTION TITLE
%-------------------------------------------------------------------------------
\cvsection{Work Experience}


%-------------------------------------------------------------------------------
%	CONTENT
%-------------------------------------------------------------------------------
\begin{cventries}

%---------------------------------------------------------
  \cventry
    {Data Engineer} % Job title
    {Axenix} % Organization
    {Санкт-Петербург, Россия} % Location
    {Авг. 2024 - Настоящее время} % Date(s)
    {
      \begin{cvitems} % Description(s) of tasks/responsibilities
        \item {Разработал модульную систему динамической генерации YAML-конфигураций для DAG'ов на основе excel маппингов, что позволило сократить время настройки новых ETL-процессов на 40\%.}
        \item {Создал комплексный набор инструментов автоматизации разработки: парсер логов Airflow (PowerShell), генератор синтетических данных, инструменты копирования артефактов и командную оболочку для управления Git-процессами, что позволило ускорить процесс адаптации новых сотрудников и улучшить качество работы.}
        \item {Оптимизировал генератор etl-потоков совместно с коллегами из других команд: внедрил шаблонизацию, асинхронную обработку и расширил покрытие генерируемых артефактов.}
        \item {Автоматизировал процесс тестирования качества данных и создал PL/pgSQL функции для согласующих тестов DataFix, значительно сократив время подготовки.}
        \item {Создал и поддерживал хранимые процедуры, функции, витрины и представления в Greenplum для операционных процессов качества данных.}
        \item {Обеспечивал интеграцию различных источников данных (ClickHouse, Oracle, Hive) и поддерживал архитектуру Data Vault 2.0.}
        \item {Организовал сообщество разработчиков в Telegram и GitLab для обмена инструментами и лучшими практиками.}
        \item {Разрабатывал etl-потоки, подготавливал релизы, хотфиксы и датафиксы.}
      \end{cvitems}
    }

%---------------------------------------------------------
  \cventry
    {Data Scientist} % Job title
    {Spacecode} % Organization
    {Краснодар, Россия (удаленно)} % Location
    {Апр. 2024 - Авг. 2024} % Date(s)
    {
      \begin{cvitems} % Description(s) of tasks/responsibilities
        \item {Разработал мультимодального ассистента на базе LLM с использованием моделей HuggingFace, RAG-подхода, LangChain и Unstructured через LLAMA\_CPP/VLLM.}
        \item {Создал систему динамического ценообразования с использованием интерпретируемых регрессионных моделей, что, по оценкам, позволит увеличить прибыль сегмента на 15\%.}
        \item {Использовал Apache Airflow для разработки ETL-пайплайна, обеспечивающего актуальность данных для ML-моделей.}
        \item {Внедрил DVC для версионирования данных моделей, что значительно упростило их отслеживание и воспроизведение результатов.}
      \end{cvitems}
    }

%---------------------------------------------------------
  \cventry
    {Data Analyst} % Job title
    {АСК «ДиректСервис»} % Organization
    {Новосибирск, Россия (удаленно)} % Location
    {Сен. 2023 - Янв. 2025} % Date(s)
    {
      \begin{cvitems} % Description(s) of tasks/responsibilities
        \item {Реализовал с нуля enterprise-level Data Warehouse (DWH) на масштабах десятков терабайт данных, используя PostgreSQL как основное хранилища и ClickHouse как аналитическое хранилище.}
        \item {Автоматизировал процесс обновления витрин данных, создав комплексный пайплайн на основе Shell-скриптов и Windows Task Scheduler.}
        \item {Настроил механизм миграции данных из ClickHouse в PostgreSQL, обеспечив бесперебойную работу аналитических систем.}
        \item {Совместно с системными аналитиками разработал методологии мониторинга и оптимизации производительности DWH.}
        \item {Настраивал мониторинг на основе расшерний Postgres, ci/cd для верифицированной доставки кода с использованием flyway и Github Actions.}
      \end{cvitems}
    }

%---------------------------------------------------------
  \cventry
    {ML Engineer} % Job title
    {ООО «ColorDent»} % Organization
    {Москва, Россия (удаленно)} % Location
    {Июн. 2023 - Дек. 2023} % Date(s)
    {
      \begin{cvitems} % Description(s) of tasks/responsibilities
        \item {Разработал ML-классификатор изображений зубов для диагностики заболеваний, используя OpenCV и TensorFlow Lite. Модель достигла точности более 95\% на тестовых данных.}
        \item {Интегрировал ML-модели в мобильное приложение с использованием MLflow для управления версиями моделей и DVC для версионирования данных.}
        \item {Оптимизировал производительность модели для работы на устройствах с ограниченными ресурсами, что позволило использовать ее на мобильных платформах без значительной потери качества.}
        \item {Разработал систему отслеживания метрик обучения моделей и внедрил автоматическое тестирование для обеспечения высокого качества решения.}
      \end{cvitems}
    }

%---------------------------------------------------------
\end{cventries}


