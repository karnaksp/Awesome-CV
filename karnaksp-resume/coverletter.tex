%!TEX TS-program = xelatex
%!TEX encoding = UTF-8 Unicode
% Awesome CV LaTeX Template for Cover Letter
%
% Этот шаблон был загружен с:
% https://github.com/posquit0/Awesome-CV
%
% Авторы:
% Claud D. Park <posquit0.bj@gmail.com>
% Lars Richter <mail@ayeks.de>
%
% Лицензия шаблона:
% CC BY-SA 4.0 (https://creativecommons.org/licenses/by-sa/4.0/)

%-------------------------------------------------------------------------------
% КОНФИГУРАЦИЯ
%-------------------------------------------------------------------------------
\documentclass[11pt, a4paper]{../awesome-cv}

% Поля страницы
\geometry{left=1.4cm, top=.8cm, right=1.4cm, bottom=1.8cm, footskip=.5cm}

% Цвет для выделения
\colorlet{awesome}{awesome-skyblue}

% Выделение секций
\setbool{acvSectionColorHighlight}{true}

% Разделитель социальных ссылок
\renewcommand{\acvHeaderSocialSep}{\quad\textbar\quad}

%-------------------------------------------------------------------------------
% ЛИЧНАЯ ИНФОРМАЦИЯ
%-------------------------------------------------------------------------------
\photo[circle,noedge,left]{./img/profile}
\name{Денис}{Ириняков}
\position{Middle Data Engineer}
\address{Россия, Санкт-Петербург}

\mobile{+7 (924) 236-02-83}
\email{irinyakov2016@yandex.ru}
\homepage{karnaksp.github.io/karnaksp}
\github{https://github.com/karnaksp}
\telegram{t.me/calmeds}

\quote{``Опытный инженер данных с глубокими знаниями в разработке и анализе данных."}

%-------------------------------------------------------------------------------
% ИНФОРМАЦИЯ О ПИСЬМЕ
%-------------------------------------------------------------------------------
\recipient
  {Спеицалисту подбора персонала}
  {Яндекс\\}
\letterdate{\today}
\lettertitle{Отклик на вакансию Data Engineer}
\letteropening{Уважаемая команда Яндекс Go,}
\letterclosing{С уважением,}
\letterenclosure[Приложено]{Резюме}
\definecolor{darktext}{HTML}{414141}
\definecolor{text}{HTML}{414141}
\definecolor{graytext}{HTML}{414141}
\definecolor{lighttext}{HTML}{999999}

%-------------------------------------------------------------------------------
\begin{document}

\makecvheader[R]

\makecvfooter
  {\today}
  {Денис Ириняков~~~·~~~Сопроводительное письмо}
  {}

\makelettertitle

%-------------------------------------------------------------------------------
% СОДЕРЖАНИЕ ПИСЬМА
%-------------------------------------------------------------------------------
\begin{cvletter}

\lettersection{Обо мне}
Я — опытный инженер данных с более чем двумя годами работы в области анализа и обработки данных. В настоящее время я работаю в Axenix на банковском проекте, где разрабатываю и оптимизирую ETL-процессы с использованием Airflow, Kafka, Hadoop, Greenplum и других технологий для ядра DWH. Ранее я работал в Spacecode и АСК «ДиректСервис», где занимался построением аналитического DWH и разработкой сервисов с поддержкой ML-моделей. Мои ключевые навыки включают работу с большими данными, ETL/ELT, SQL, Python и DevOps практиками.

\lettersection{Почему ваша компания?}
Яндекс — лидер в области технологий и инноваций в России и за ее пределами. Ваша компания славится передовым подходом к обработке данных и машинному обучению, что идеально соответствует моим профессиональным целям. Меня вдохновляют метариалы компании Яндекс на конференциях и докладах, очень интересно было послушать доклады на HighLoad о YTsaurus и Clickhouse. Детализация собственных разработок и компетентность специалистов сильно впечатляет, что мотивирует развиваться и поддерживает степень вовлеченности в проекты. 

\lettersection{Почему я?}
Мой опыт работы с ETL/ELT-процессами на Airflow, dbt, построением DWH на PostgreSQL и Greenplum, оптимизацией запросов и работы СУБД отлично Вам подходит. На протяжении обучения и работы я постоянно углубляюсь в детали архитектуры, стараюсь постигать и использовать преимущества разных языков программирования и инструментов. Я стараюсь соединять DevOps практики и классические инструменты дата инженера, поскольку понимаю их важность в работе команды, и, если синтез алгоритмов программирования и etl подарил нам MapReduce 1.0, то синтез кубера и MapReduce подарил YAMR или YARN и Spark и в общем-то еще много незаменимых сейчас технологий.

\end{cvletter}

\makeletterclosing

\end{document}
